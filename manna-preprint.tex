\documentclass[10pt,twocolumn,fleqn]{article}
\usepackage{geometry}
\geometry{letterpaper,margin=1.5cm}

\usepackage[parfill]{parskip}
\usepackage[T1]{fontenc}
\usepackage[utf8]{inputenc}
\usepackage{authblk}

\usepackage{tikz, tkz-graph}
\usepackage{pgfplots}
\usepackage{pgfcb}
\pgfplotsset{
  every axis legend/.append style={draw=none, fill=black!2, cells = {anchor=west}, font = \small},
  every axis grid/.append style = {color=black!60, dotted}
}

\usepackage[natbib=true,doi=false,url=false,isbn=false,backend=bibtex]{biblatex}
\bibliography{/home/tpoisot/texmf/bibtex/bib/local/library.bib}

\title{The dynamics of neutral and trait-based multi-trophic interactions}
\date{}

\author[1,2]{Timoth\'ee Poisot}
\author[1,2]{Dominique Gravel}

\affil[1]{Universit\'e du Qu\'ebec \`a Rimouski, Rimouski, Canada.}
\affil[2]{Quebec Centre for Biodiversity Science, McGill University, Montr\'eal, Canada.}
\affil[*]{email: \texttt{timothee\_poisot@uqar.ca}}

\begin{document}

\maketitle

\begin{abstract}
We describe \emph{manna}, a simple modeling framework of individuals
establishing trophic interactions under neutral or trait-based rules. We use
this model to show that neutral and trait-based interactions differ in their
temporal dynamics, and discuss how our results contribute to highlighting the
need to understand spatial and temporal dynamics of species interaction
networks.
\end{abstract}

\section{Introduction}

Over the last years, there has been the emergence of two bodies of theory
regarding the structure of ecological networks. On one hand, proponents of the
``niche'' approach emphasized the importance of trait matching between species
\citep{olesen_missing_2011}. To simplify, two species can interact assuming
that their traits allow it, e.g.  the jaw morphometry of the predator allows it
to bite, and eventually eat, its chosen prey. Arguably, this has been the
dominant view in the field of network studies, and it inspired many
methodological tools which peformed well at reproducing the structure of food
webs \citep{williams_simple_2000}. On the other hand, several authors adopted a
``neutral'' perspective, putting more emphasis on the fact that interactions
rely on density-dependent processes, most notably the probability that
individuals from the prey and predator species will meet. Although this
perspective was originally develloped in plant--mutualistic interactions
\citep{bluthgen_measuring_2006}, where it showed a great predictive power,
recent work expanded it to food webs with equal success
\citep{canard_emergence_2012}. Much like in other fields of community ecology,
there is a need to integrate these two perspectives so as to derive more
accurate predictions about the structure of networks, and their variability
through space and time.

Indeed, the basic ``ingredients'' of both the neutral (population sizes) and
trait-based (trait values carried by populations or individuals) are expected to
vary through space and time. Not only will they vary because of extrinsic
constraints (e.g. organisms with different body sizes are differently affected
by environmental disturbances), but they will also vary because the populations
are interacting. Ecological dynamics results in covariances in the population
sizes of preys and predators, and coevolutionary feedbacks will select different
traits values. For this reason, accounting for both the neutral and trait-based
effects will have two advantages. First, it will somehow reconcile the two
perspectives on network structure, and allow a conceptual unification of the two
most important families of explanations of network structure. Second, it will
integrate the basal processes involved in these perspectives, and thus increase
the accuracy of our expectations regarding network structure and dynamics.
Because networks are notoriously difficult to adequately sample, and replicating
them over time and space is not easily done, there is value in exploring the
behavior of simple, highly phenomenological models.

In this manuscript, we present such a model, whose goal it is to account for
both trait-based and neutral processes in the assembly, structure, and variation
of multi-trophic networks. We add a layer of individual based population
dynamics on top of the well-known ``niche model'', and use this to derive
predictions about (1) the structure of multi-trophic communities under different
scenarios, and (2) the variation of network structure over time. We show that
although there is a clear effect of switching from neutral to trait-based rules,
the entirely neutral situation cannot be differenciated from a trait-based
networks in which predators have a large feeding range. We discuss the
implications of these findings, and how they can help refine future research
questions pertaining to the relationships between neutral and trait-based
processes.

\section{The model}

We define a time-discrete, individual based model of multi-trophic interactions
in a single patch. Species are defined by a vector of functional traits, them
being their niche position ($\mathbf{n}_i$), the centroid of their feeding range
($\mathbf{c}_i$), the breadth of their range ($\mathbf{r}_i$), and their maximal
population size ($\mathbf{K}_i$).

\subsection{Generation of the species pool}

The species pool is generated so as to obtain a number of species $S$, with an
expected connectance value of $Co$. To generate the species pool, we draw at
random in an uniform distribution ranging from 0 to 1, $S$ values forming the
vector $\mathbf{n}$. The vectors $\mathbf{c}$ and $\mathbf{r}$ are generated
following the method described by \citet{williams_simple_2000}. The number of
individuals in each species is $\mathbf{p}_i$, and is drawn at random according
to the following procedure.

First, the values of $\mathbf{K}_i$ are drawn at random from the
uniform distribution between 0 and $K_{\mathrm{max}}(i)$, where
$K_{\mathrm{max}}(i)$ is simply $10^3\times(1-\mathbf{n}_i)+100$. This way,
species with the large $\mathbf{n}_i$ have lower carrying capacity, and
species with low $\mathbf{n}_i$ have high carrying capacity. This accounts for
the well known negative relationship between trophic rank and population size,
and give a structure to the community.
To initialize the simulation, a number of individuals $\mathbf{p}_i$ is drawn
from the uniform distribution from $10$ to $\mathbf{K}_i$.

\subsection{Simulation}

Before the interactions happen and the demographic changes are calculated,
there is a (possible) immigration step, to implicitely simulate some spatial
context and maintain species richness. The identity of the species receiving
each migrant is drawn at random from within the regional species pool. This
makes it so that locally extinct species can be rescued through immigration.
After the immigration is over, the probability that each individual will
reproduce or die is calculated given the following set of rules. The mortality
probability of individual $i$ is

\begin{equation}
	m_i = \mu+s_ic_\mu
\end{equation}

\noindent, while its natality probability is 

\begin{equation}
	b_i = \nu-s_ic_\nu
\end{equation}

\noindent, and $s_i$ is a scaling factor related to the niche position, wherein

\begin{equation}
	1-(1-\mathbf{n}_i)^k
\end{equation}

\noindent. The exponent $k$ regulates the strength of the scaling. The
population size of species $i$ at time $t+1$ is calculated as

\begin{equation}
	\mathrm{min}(\mathbf{K}_i, \mathbf{p}_i - \mathcal{B}(m_i, \mathbf{p}_i) + \mathcal{B}(b_i, \mathbf{p}_i))
\end{equation}

\noindent, where $\mathcal{B}(p,n)$ is the number of successes of a Bernoulli
process with probability of success of $p$ and $n$ trials.

After the demographic changes are done, we pick at random 15\% of the
individuals, which will be seeking a prey at this timestep (decreasing or
increasing this fraction will, respectively, simulates an increase or decrease
in the prey handling time). Each indiviual has an equal probability of being
picked, so that species with high abundances will be over-represented among the
pools of predators (in this regard, our model assumes that the search for preys
is an active task, which is a fairly common situation in nature). The identity
of preys items is detailed in the next two sub-sections. The interactions are
done on an individual basis, meaning that we pick a predator, then its prey, and
reproduce this routine once we have reached enough predators.

\subsection{Neutral interactions}

In the neutral scenario, the only factor regulating the identity of the prey
is its abundance, i.e. none of the traits are looked at when deciding if the
predator can consume the prey. Once the predating individual is selected, we pick at random
one individual from any species (including the possibility of cannibalism) in
the pool. The predated individual is then removed. In this scenario, it is
possible that a small species will eat a much larger species, though it should
be kept in mind that because of allometric scaling between the body size traits
and demographic parameters, we expect the interactions to still retain some size
structure. Because of this feature, our model departs from the purely neutral
assumption made by \citet{canard_emergence_2012}.

\subsection{Trait-based interactions}

In the trait-based scenario, preys are selected only among the species that
can be fed upon, as in the \emph{niche} model of \citet{williams_simple_2000}.
Preys are still selected on the basis of their abundances, although this time we
consider the relative abundance among all preys falling within the predator
feeding range.

\begin{figure}[tb]
	\begin{center}
		\definecolor{COLOR0}{rgb}{0.4, 0.4, 0.4}
\definecolor{COLOR3}{rgb}{0.0,0.0,0.0}
\pgfdeclarelayer{background}
\pgfdeclarelayer{foreground}
\pgfsetlayers{background,main,foreground}
\begin{tikzpicture}[scale=2.5]
\node at (-0,1.2) [circle, line width=1, fill=COLOR0,  inner sep=0pt, minimum size = 5pt,] (4) {};
\node at (-0.3526712,0.1145898) [circle, line width=1, fill=COLOR0,  inner sep=0pt, minimum size = 5pt,] (8) {};
\node at (-0.5706339,0.4145898) [circle, line width=1, fill=COLOR0,  inner sep=0pt, minimum size = 5pt,] (2) {};
\node at (0.3526712,1.0854102) [circle, line width=1, fill=COLOR0,  inner sep=0pt, minimum size = 5pt,] (7) {};
\node at (-0.5706339,0.7854102) [circle, line width=1, fill=COLOR0,  inner sep=0pt, minimum size = 5pt,] (1) {};
\node at (-0.3526712,1.0854102) [circle, line width=1, fill=COLOR0,  inner sep=0pt, minimum size = 5pt,] (5) {};
\node at (0,0) [circle, line width=1, fill=COLOR0,  inner sep=0pt, minimum size = 5pt,] (3) {};
\node at (0.5706339,0.4145898) [circle, line width=1, fill=COLOR0,  inner sep=0pt, minimum size = 5pt,] (6) {};
\node at (0.3526712,0.1145898) [circle, line width=1, fill=COLOR0,  inner sep=0pt, minimum size = 5pt,] (15) {};
\node at (0.5706339,0.7854102) [circle, line width=1, fill=COLOR0,  inner sep=0pt, minimum size = 5pt,] (16) {};
\begin{pgfonlayer}{background}
\tikzset{EdgeStyle/.style = {->,shorten >=1pt,>=stealth, bend right=2}}
\tikzset{EdgeStyle/.style = {-, shorten >=1pt, >=stealth, bend right=2, line width=0.5, color=COLOR3}}
\Edge (4)(5)
\Edge (4)(8)
\Edge (8)(2)
\Edge (2)(5)
\Edge (2)(7)
\Edge (1)(3)
\Edge (1)(5)
\Edge (1)(6)
\Edge (5)(8)
\Edge (3)(6)
\Edge (3)(7)
\Edge (3)(8)
\Edge (15)(2)
\Edge (15)(3)
\Edge (16)(1)
\Edge (16)(4)
\Edge (16)(7)
\end{pgfonlayer}
\end{tikzpicture}
		\hskip 1em%
		\definecolor{COLOR0}{rgb}{0.4, 0.4, 0.4}
\definecolor{COLOR3}{rgb}{0.8,0.8,0.8}
\pgfdeclarelayer{background}
\pgfdeclarelayer{foreground}
\pgfsetlayers{background,main,foreground}
\begin{tikzpicture}[scale=2.5]
\node at (-0,1.2) [circle, line width=1, fill=COLOR0,  inner sep=0pt, minimum size = 5pt,] (4) {};
\node at (-0.3526712,0.1145898) [circle, line width=1, fill=COLOR0,  inner sep=0pt, minimum size = 5pt,] (8) {};
\node at (-0.5706339,0.4145898) [circle, line width=1, fill=COLOR0,  inner sep=0pt, minimum size = 5pt,] (2) {};
\node at (0.3526712,1.0854102) [circle, line width=1, fill=COLOR0,  inner sep=0pt, minimum size = 5pt,] (7) {};
\node at (-0.5706339,0.7854102) [circle, line width=1, fill=COLOR0,  inner sep=0pt, minimum size = 5pt,] (1) {};
\node at (-0.3526712,1.0854102) [circle, line width=1, fill=COLOR0,  inner sep=0pt, minimum size = 5pt,] (5) {};
\node at (0,0) [circle, line width=1, fill=COLOR0,  inner sep=0pt, minimum size = 5pt,] (3) {};
\node at (0.5706339,0.4145898) [circle, line width=1, fill=COLOR0,  inner sep=0pt, minimum size = 5pt,] (6) {};
\node at (0.3526712,0.1145898) [circle, line width=1, fill=COLOR0,  inner sep=0pt, minimum size = 5pt,] (15) {};
\node at (0.5706339,0.7854102) [circle, line width=1, fill=COLOR0,  inner sep=0pt, minimum size = 5pt,] (16) {};
\begin{pgfonlayer}{background}
\tikzset{EdgeStyle/.style = {->,shorten >=1pt,>=stealth, bend right=2}}
\tikzset{EdgeStyle/.style = {-, shorten >=1pt, >=stealth, bend right=2, line width=0.1, color=COLOR3, dashed}}
\Edge (3)(7)
\Edge (3)(8)
\Edge (2)(7)
\Edge (16)(1)
\Edge (1)(3)
\Edge (1)(5)
\tikzset{EdgeStyle/.style = {-, shorten >=1pt, >=stealth, bend right=2, line width=0.5, color=black}}
\Edge (4)(5)
\Edge (4)(8)
\Edge (8)(2)
\Edge (2)(5)
\Edge (1)(6)
\Edge (5)(8)
\Edge (3)(6)
\Edge (15)(2)
\Edge (15)(3)
\Edge (16)(4)
\Edge (16)(7)
\end{pgfonlayer}
\end{tikzpicture}
	\end{center}
	\caption{Comparison of the structure of networks under the neutral (left) and trait-based (right) approaches. When the only rule for the existence of an interaction is that the two populations are large enough to interact (neutral rules), we expect more interactions than when traits are used to delineate a feeding range (trait-based rules). These trait-forbidden links are dashed in the right network. Switching from one situation to the other can lead to complex effects, both on networks dyamics and community composition.}
	\label{fig:nitr}
\end{figure}

\subsection{Implementation and availability}

A C99 implementation of the model, using the Gnu Scientific Library, is
available at \texttt{https://github.com/tpoisot/ms\_pop\_networks/}, under the
conditions of the GNU GPL licence. We would appreciate that a reference to this
preprint is made when the program is used. A R script to reproduce the results
showed in this paper is also made available.

\section{Methods}

\subsection{Simulations}

We used the model to generate species pools of size $S = 50$, with connectance
varying from $0.01$ to $0.5$ by varying increments choosen to represent the
changes induced by connectance. 5 individuals of any random species migrated
into the system at each timestep. Each community is simulated under the neutral
and trait based assumptions, and replicated 20 times (20 times for the neutral
scenario, and 20 times for each value of connectance under the trait-based
scenario). Simulations are done over $5\times 10^3$ timesteps, as preliminary
analyses showed it was long enough to reach an equilibrium in number of
persisting species, number of persisting links, and overtime dissimilarity.

\subsection{Analyses}

Networks were aggregated in bins of 10 timesteps for the analysis
(exploratory work showed that the bin size made no quantitative difference on
the results, though it did considerably speed up the calculations). Within
each bin, we measure the number of species in the network ($\mathcal{S}$), the
number of links ($\mathcal{L}$), the effective connectance ($\mathcal{C}$, expressed as
$\mathcal{L}/\mathcal{S}^2$).

Within each run of the simulation, we measure components of network
dissimilarity through time, using the method described in
\citet{poisot_dissimilarity_2012}. We measure turnover of interactions and
community structure between the network at time $t$ and the network at time
$t+1$. We report the overal network dissimilarity ($\beta_{WN}$), the
dissimilarity of links across shared species ($\beta_{OS}$), the species
turnover ($\beta_{S}$), and its contribution to network turnover
($\beta_{\mathrm{contrib}}$). In addition, we report the turnover of predators
($\beta_{U}$) and preys ($\beta_{L}$) species through time.

\section{Results}

\subsection{Community dynamics}

\begin{figure}[tb]
	\begin{center}
		\begin{tikzpicture} %% Part 1: Species
			\begin{axis}[height=5cm, width=0.9\columnwidth,
			xmin = 0, xmax = 2500, xlabel=Timestep,
			ymin = 0, ymax = 50, ylabel=Number of unique species,
			grid=major]
			\addplot+[mark=none,solid,color=black,very thick] table [x=t,y=sp] {./figures/dyn-0.1.neutral.dat};
			\addlegendentry{Neutral};
			\addplot+[mark=none,solid,color=Spectral1,thick] table [x=t,y=sp] {./figures/dyn-0.01.trait.dat};
			\addlegendentry{$Co = 1\times10^{-2}$};
			\addplot+[mark=none,solid,color=Spectral3,thick] table [x=t,y=sp] {./figures/dyn-0.05.trait.dat};
			\addlegendentry{$Co = 5\times10^{-2}$};
			\addplot+[mark=none,solid,color=Spectral5,thick] table [x=t,y=sp] {./figures/dyn-0.1.trait.dat};
			\addlegendentry{$Co = 1\times10^{-1}$};
			\addplot+[mark=none,solid,color=Spectral7,thick] table [x=t,y=sp] {./figures/dyn-0.2.trait.dat};
			\addlegendentry{$Co = 2\times10^{-1}$};
			\addplot+[mark=none,solid,color=Spectral9,thick] table [x=t,y=sp] {./figures/dyn-0.5.trait.dat};
			\addlegendentry{$Co = 5\times10^{-1}$};
			\end{axis}
		\end{tikzpicture}
		\begin{tikzpicture} %% Part 2: Links
			\begin{axis}[height=5cm, width=0.9\columnwidth,
			xmin = 0, xmax = 2500, xlabel=Timestep,
			ymin = 0, ymax = 80, ylabel=Number of interactions,
			grid=major]
			\addplot+[mark=none,solid,color=black,very thick] table [x=t,y=li] {./figures/dyn-0.1.neutral.dat};
			\addplot+[mark=none,solid,color=Spectral1,thick] table  [x=t,y=li] {./figures/dyn-0.01.trait.dat};
			\addplot+[mark=none,solid,color=Spectral3,thick] table  [x=t,y=li] {./figures/dyn-0.05.trait.dat};
			\addplot+[mark=none,solid,color=Spectral5,thick] table  [x=t,y=li] {./figures/dyn-0.1.trait.dat};
			\addplot+[mark=none,solid,color=Spectral7,thick] table  [x=t,y=li] {./figures/dyn-0.2.trait.dat};
			\addplot+[mark=none,solid,color=Spectral9,thick] table  [x=t,y=li] {./figures/dyn-0.5.trait.dat};
			\end{axis}
		\end{tikzpicture}
		\begin{tikzpicture} %% Part 3: Connectance
			\begin{axis}[height=5cm, width=0.9\columnwidth,
			xmin = 0, xmax = 2500, xlabel=Timestep,
			ymin = 0, ymax = 0.4, ylabel=Connectance,
			grid=major]
			\addplot+[mark=none,solid,color=black,very thick] table [x=t,y=co] {./figures/dyn-0.1.neutral.dat};
			\addplot+[mark=none,solid,color=Spectral1,thick] table  [x=t,y=co] {./figures/dyn-0.01.trait.dat};
			\addplot+[mark=none,solid,color=Spectral3,thick] table  [x=t,y=co] {./figures/dyn-0.05.trait.dat};
			\addplot+[mark=none,solid,color=Spectral5,thick] table  [x=t,y=co] {./figures/dyn-0.1.trait.dat};
			\addplot+[mark=none,solid,color=Spectral7,thick] table  [x=t,y=co] {./figures/dyn-0.2.trait.dat};
			\addplot+[mark=none,solid,color=Spectral9,thick] table  [x=t,y=co] {./figures/dyn-0.5.trait.dat};
			\end{axis}
		\end{tikzpicture}
	\end{center}
	\caption{Temporal dynamics of network structure, as a function of wether the system behaves neutrally (black lines) or under trait-based rules (colored lines). The first 2500 timesteps are shown, all simulations reached equilibrium at approx. $t = 1000$. \textbf{A.} Number of unique species ($\mathcal{S}$). \textbf{B.} Number of unique interactions ($\mathcal{L}$) \textbf{C.} Connectance of the network ($\mathcal{C}$).}
	\label{f:comdyn}
\end{figure}

Regardless of the rules or connectance, all simulation conditions yield a common
pattern: a rapid decrease is the number of species (\emph{i.e.} sorting) and
links (Fig.~\ref{f:comdyn}). This results in changes in connectance, with all
the simulations eventually stabilizing around a value of $\mathcal{C}\approx
0.05$. The number of species maintained is higher in the neutral conditions.
When connectance increases, the number of persisting species increasing as well.
For a connectance of $\mathrm{Co} = 0.5$, there are almost no differences in
species richness between the neutral and trait-based situations. The same is
true of the number of interactions. The equal connectance across the different
rules is easily explained by the fact that, with increased restrictions on trait
matching (i.e. switching from neutral to trait-based, or from high connectance
to lower connectance within trait-based rules), both the number of species and
the number of interactions decreased. However, the dynamics of connectance
through time were different. Neutral systems started with high realized
connectance (on average 0.3, close to the values reported by
\citet{canard_emergence_2012} for the purely neutral scenario), and the sharply
decreased. On the other hand, trait-based systems generated with a low
connectance remained at this level through time.

\subsection{Networks dynamics}

\begin{figure}[tb]
	\begin{center}
		\begin{tikzpicture} %% Part 1: whole network
			\begin{axis}[height=5cm, width=0.9\columnwidth,
			xmin = 0, xmax = 2500, xlabel=Timestep,
			ymin = 0, ymax = 1, ylabel=Whole network dissimilarity,
			grid=major]
			\addplot+[mark=none,solid,color=black,very thick] table [x=t,y=WN] {./figures/beta-0.1.neutral.dat};
			\addlegendentry{Neutral};
			\addplot+[mark=none,solid,color=Spectral1,thick] table [x=t,y=WN] {./figures/beta-0.01.trait.dat};
			\addlegendentry{$Co = 1\times10^{-2}$};
			\addplot+[mark=none,solid,color=Spectral3,thick] table [x=t,y=WN] {./figures/beta-0.05.trait.dat};
			\addlegendentry{$Co = 5\times10^{-2}$};
			\addplot+[mark=none,solid,color=Spectral5,thick] table [x=t,y=WN] {./figures/beta-0.1.trait.dat};
			\addlegendentry{$Co = 1\times10^{-1}$};
			\addplot+[mark=none,solid,color=Spectral7,thick] table [x=t,y=WN] {./figures/beta-0.2.trait.dat};
			\addlegendentry{$Co = 2\times10^{-1}$};
			\addplot+[mark=none,solid,color=Spectral9,thick] table [x=t,y=WN] {./figures/beta-0.5.trait.dat};
			\addlegendentry{$Co = 5\times10^{-1}$};
			\end{axis}
		\end{tikzpicture}\hfill%
		\begin{tikzpicture} %% Part 2: OS
			\begin{axis}[height=5cm, width=0.9\columnwidth,
			xmin = 0, xmax = 2500, xlabel=Timestep,
			ymin = 0, ymax = 1, ylabel=Common species dissimilarity,
			grid=major]
			\addplot+[mark=none,solid,color=black,very thick] table [x=t,y=OS] {./figures/beta-0.1.neutral.dat};
			\addplot+[mark=none,solid,color=Spectral1,thick] table  [x=t,y=OS] {./figures/beta-0.01.trait.dat};
			\addplot+[mark=none,solid,color=Spectral3,thick] table  [x=t,y=OS] {./figures/beta-0.05.trait.dat};
			\addplot+[mark=none,solid,color=Spectral5,thick] table  [x=t,y=OS] {./figures/beta-0.1.trait.dat};
			\addplot+[mark=none,solid,color=Spectral7,thick] table  [x=t,y=OS] {./figures/beta-0.2.trait.dat};
			\addplot+[mark=none,solid,color=Spectral9,thick] table  [x=t,y=OS] {./figures/beta-0.5.trait.dat};
			\end{axis}
		\end{tikzpicture}
		\begin{tikzpicture} %% Part 3: S
			\begin{axis}[height=5cm, width=0.9\columnwidth,
			xmin = 0, xmax = 2500, xlabel=Timestep,
			ymin = 0, ymax = 1, ylabel=Species dissimilarity,
			grid=major]
			\addplot+[mark=none,solid,color=black,very thick] table [x=t,y=S] {./figures/beta-0.1.neutral.dat};
			\addplot+[mark=none,solid,color=Spectral1,thick] table  [x=t,y=S] {./figures/beta-0.01.trait.dat};
			\addplot+[mark=none,solid,color=Spectral3,thick] table  [x=t,y=S] {./figures/beta-0.05.trait.dat};
			\addplot+[mark=none,solid,color=Spectral5,thick] table  [x=t,y=S] {./figures/beta-0.1.trait.dat};
			\addplot+[mark=none,solid,color=Spectral7,thick] table  [x=t,y=S] {./figures/beta-0.2.trait.dat};
			\addplot+[mark=none,solid,color=Spectral9,thick] table  [x=t,y=S] {./figures/beta-0.5.trait.dat};
			\end{axis}
		\end{tikzpicture}\hfill%
		\begin{tikzpicture} %% Part 4: contrib
			\begin{axis}[height=5cm, width=0.9\columnwidth,
			xmin = 0, xmax = 2500, xlabel=Timestep,
			ymin = 0, ymax = 1, ylabel=Contribution of sp. turnover,
			grid=major]
			\addplot+[mark=none,solid,color=black,very thick] table [x=t,y=contrib] {./figures/beta-0.1.neutral.dat};
			\addplot+[mark=none,solid,color=Spectral1,thick] table  [x=t,y=contrib] {./figures/beta-0.01.trait.dat};
			\addplot+[mark=none,solid,color=Spectral3,thick] table  [x=t,y=contrib] {./figures/beta-0.05.trait.dat};
			\addplot+[mark=none,solid,color=Spectral5,thick] table  [x=t,y=contrib] {./figures/beta-0.1.trait.dat};
			\addplot+[mark=none,solid,color=Spectral7,thick] table  [x=t,y=contrib] {./figures/beta-0.2.trait.dat};
			\addplot+[mark=none,solid,color=Spectral9,thick] table  [x=t,y=contrib] {./figures/beta-0.5.trait.dat};
			\end{axis}
		\end{tikzpicture}
	\end{center}
	\caption{Temporal dynamics of network turnover, as a function of wether the system behaves neutrally (black lines) or under trait-based rules (colored lines). The first 2500 timesteps are shown, all simulations reached equilibrium at approx. $t = 1000$. \textbf{A.} Whole-network dissimilarity ($\beta_{WN}$) remains high under the neutral scenario. \textbf{B.} The same is true of interaction turnover across shared species, $\beta_{OS}$. \textbf{C.} Species turnover displays a hump-shaped pattern with connectance (see also {\color{red}FIGURE XXX}). \textbf{D.} The impact of species turnover on interactions turnover is low is the neutral model, but increasingly higher when trait-based rules are used and connectance decreases.}
	\label{f:netdyn}
\end{figure}

Temporal network dynamics was strongly affected by whether the system behaves
neutrally or not, and by the fondamental connectance from the species pool in
the trait-based simulations. As expected, neutral networks had a high turnover,
and a low impact of species turnover on network structure turnover. Comparing
the different scenarios make sense as the amount of temporal species turnover is
remarkably similar in all of them (Fig.~\ref{f:netdyn}). Under the trait-based
simulations, decreasing connectance decreased the amount of temporal turnover in
network structure. These results emphasize that networks in which trait matching
have a strong impact, and predators tend to have narrow niches, display more
temporal stability.

\section{Discussion}

One interesting outcome of the simulations presented here is that neutral and
trait-based systems differed on some, but not all, of the features of the
networks they generated. Increased restrinctions on the range of preys that can
be consumed (through the generation of networks with decreasing expected
connectances) resultes in less species and less interactions being maintained at
equilibrium. However, because the loss of species and interactions happenned at
a similar rythm with regard to changes in expected connectances, the networks
showed the same realized connectance at equilibrium. In our system at least,
there is no signature of neutrality or traits on network connectance. Looking at
the dynamics of network turnover, however, is far more telling about which
mechanisms may be acting in a particular system. Increasingly neutral networks
had a lot more temporal turnover, both in terms of their overall structure, or
the interactions between shared species, than did trait-based networks with low
expected connectance. The temporal dynamics of species dissimilarity is less
obvious, though networks with intermediate expected connectances tended to have
more species turnover. The more revealing difference, however, is in the impact
of species turnover on network turnover. While it is almost null in neutral
networks, or in networks with high expected connectance, it moves closer to
unity when expected connectance decreases. This results makes sense, in that
when  the expected connectance is low, species have a low generality
\citep{schoener_food_1989}, and there is very low functional redundancy across
species.

These results highlight that determining wether network structure is driven by
neutral or trait-based dynamics may require a more dynamic approach than what is
currently done. Although previous studies focused on stastitical analsyses to
detect whether relative abundances, or traits combinations, allowed predicting
the existence of interactions, we show that these two factors can interact in
complex ways. Notably, we report no impact on network connectance, with
highly-connected systems being impossible to distinguish from neutral systems on
the measures we followed. However, by following network dynamics, one can paint
a sharper picture of the different facets of network turnover, all of which
respond strongly to the relative importance of neutral and trait-based
processes. Based on this information, we suggest that follwoing the dynamics of
networks, however hard it may be to achieve in nature, can be an invaluable
information in order to pinpoint which mechanisms are driving the structure of
this network.

Our results emphasize that connectance (in this case, fondamental connectance in
the species pool, akin to the niche breadth of the predators) is extremely
important in driving the temporal dynamics of network structure. Networks in
which the connectance is high (i.e. generalist predators) behave almost
neutrally, as they will pick prey items within their large range based on their
abundances. On the other hands, systems with mostly specialist predators will be
more consistent in their structure through time, because interactions will
happen between the same restricted pool of species. So far, it is important to
point out that the literrature on neutral and trait-based effects on network
structure assumed that the whole system behaves either neutrlly or not. As our
results illustrate, neutral effects are likely to happen within the range (i.e.
palatable prey items will be picked as a function of their relative abundances).
When there are strong difference in niche breadth among predators, this opens
the interesting possiblity of understanding the contribution of each species to
network turnover. Generalist species could behave more neutrally, i.e. respond
more strongly to variations in prey abundances, than will specialist species. By
quantifying (i) how much of neutrality and trait matching is involved in the
variation of the overall network structure, and (ii) how these effects can be
partitionned at the species level, it is extremely likely that we will improve
our understanding of the dynamics of interaction networks.

\textbf{Acknowledgements}: We are grateful to the participants of the
``Gradient-based ecological research'' working group from the Santa Fe Institute
for stimulating discussions.

\printbibliography

\end{document}
